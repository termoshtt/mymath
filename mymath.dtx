% \CheckSum{0}
% \iffalse
%<package>\NeedsTeXFormat{LaTeX2e}
%<package>\ProvidesPackage{mymath}[2011/09/06 ver0.1]
%<*driver>
\documentclass[a4paper]{ltxdoc}
\usepackage{amsmath}
\usepackage{mymath}
\setcounter{StandardModuleDepth}{1}
\GetFileInfo{mymath.sty}
\begin{document}
\DocInput{mymath.dtx}
\end{document}
%</driver>
% \fi
%
% \title{mymath : 自分用数学マクロ集}
% \author{寺村 俊紀}
% \maketitle
%
% 自分用の数学用マクロ集。
% 
% \StopEventually{}
%
% \begin{macro}{\mathvc}
% ベクトル記号を作る(bmパッケージが使えないとき)
%    \begin{macrocode}
%<*package>
\newcommand{\mathvc}[1]{\makebox{\boldmath$#1$}}
%</package>
%    \end{macrocode}
% \end{macro}
%
% \begin{macro}{\erf}
%    誤差関数$\erf$
%    \begin{macrocode}
%<*package>
\newcommand{\erf}{\text{erf}}
%</package>
%    \end{macrocode}
% \end{macro}
%
%
% \begin{macro}{\dif}
% 常微分記号$\dif{x}$を出力する($x$は引数)
%    \begin{macrocode}
%<*package>
\newcommand{\dif}[1]{\frac{d}{d #1 } }
%</package>
%    \end{macrocode}
% \end{macro}
%
% \begin{macro}{\difn}
% 常微分記号$\difn{x}{n}$を出力する($x$は第1引数、$n$は第2引数)
%    \begin{macrocode}
%<*package>
\newcommand{\difn}[2]{\frac{d^#2}{d #1^#2 } }
%</package>
%    \end{macrocode}
% \end{macro}
%
% \begin{macro}{\diff}
% 常微分記号$\diff{x}{u}$を出力する($x$は第1引数、$u$は第2引数)
%    \begin{macrocode}
%<*package>
\newcommand{\diff}[2]{\frac{d #2 }{d #1 } }
%</package>
%    \end{macrocode}
% \end{macro}
% 
% \begin{macro}{\diffn}
% 常微分記号$\diffn{x}{u}{n}$を出力する($x$は第1引数、$u$は第2引数、$n$は第3引数)
%    \begin{macrocode}
%<*package>
\newcommand{\diffn}[3]{\frac{d^#3 #2 }{d #1^#3 } }
%</package>
%    \end{macrocode}
% \end{macro}
%
% \begin{macro}{\difp}
% 偏微分記号$\difp{x}$を出力する($x$は引数)
%    \begin{macrocode}
%<*package>
\newcommand{\difp}[1]{\frac{\partial}{\partial #1 } }
%</package>
%    \end{macrocode}
% \end{macro}
%
% \begin{macro}{\difpn}
% 偏微分記号$\difpn{x}{n}$を出力する($x$は引数)
%    \begin{macrocode}
%<*package>
\newcommand{\difpn}[2]{\frac{\partial^#2}{\partial #1^#2 } }
%</package>
%    \end{macrocode}
% \end{macro}
%
% \begin{macro}{\diffp}
% 偏微分記号$\diffp{x}{u}$を出力する($x$は第1引数、$u$は第2引数)
%    \begin{macrocode}
%<*package>
\newcommand{\diffp}[2]{\frac{\partial #2 }{\partial #1 } }
%</package>
%    \end{macrocode}
% \end{macro}
% 
% \begin{macro}{\diffpn}
% 偏微分記号$\diffpn{x}{u}{n}$を出力する($x$は第1引数、$u$は第2引数)
%    \begin{macrocode}
%<*package>
\newcommand{\diffpn}[3]{\frac{\partial^#3 #2 }{\partial #1^#3 } }
%</package>
%    \end{macrocode}
% \end{macro}
% 
% \begin{macro}{\KSE}
% 蔵本-Sivashinsky方程式\\
% \[ \KSE \]
%    \begin{macrocode}
%<*package>
\newcommand{\KSE}{ \diffp{t}{u} + u \diffp{x}{u} + \diffpn{x}{u}{2} + \diffpn{x}{u}{4} = 0 }
%</package>
%    \end{macrocode}
% \end{macro}
%
% \begin{macro}{\KSEtw}
% 蔵本-Sivashinsky方程式 定常進行波解\\
% \[ \KSEtw \]
%    \begin{macrocode}
%<*package>
\newcommand{\KSEtw}{ \diffp{t}{u} + \left(u -c \right) \diffp{x}{u} + \diffpn{x}{u}{2} + \diffpn{x}{u}{4} = 0 }
%</package>
%    \end{macrocode}
% \end{macro}
%
% \begin{macro}{\SHE}
% Swift-Hohenberg方程式\\
% \[ \SHE\]
%    \begin{macrocode}
%<*package>
\newcommand{\SHE}{ \diffp{t}{u} = \left( r - \left( \nabla^2 + q_c^2 \right)^2 \right) u + b_3 u^3 - b_5 u^5 }
%</package>
%    \end{macrocode}
% \end{macro}
%
% \begin{macro}{\SHEd}
% 一次元Swift-Hohenberg方程式\\
% \[ \SHEd\]
%    \begin{macrocode}
%<*package>
\newcommand{\SHEd}{ \diffp{t}{u} = \left( r - \left( \difpn{x}{2} + q_c^2 \right)^2 \right) u + b_3 u^3 - b_5 u^5 }
%</package>
%    \end{macrocode}
% \end{macro}
%
% \Finale
