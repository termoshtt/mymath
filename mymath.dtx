% \CheckSum{0}
% \iffalse
%<package>\NeedsTeXFormat{LaTeX2e}
%<package>\ProvidesPackage{mymath}[2011/09/06 ver0.1]
%<*driver>
\documentclass[a4paper]{ltxdoc}
\usepackage{amsmath}
\usepackage{amssymb}
\usepackage{color}
\usepackage{mymath}
\setcounter{StandardModuleDepth}{1}
\GetFileInfo{mymath.sty}
\begin{document}
\DocInput{mymath.dtx}
\end{document}
%</driver> % \fi
%
% \title{mymath : 数学マクロ集}
% \author{寺村 俊紀}
% \maketitle
%
% 使用にはamsmath,amssymb,colorが必要。
% 
% \StopEventually{}
%
% \begin{macro}{\mathvc}
%    ベクトル記号を作る(bmパッケージが使えないとき)
%    \begin{macrocode}
\newcommand{\mathvc}[1]{\makebox{\boldmath$#1$}}
%    \end{macrocode}
% \end{macro}
%
% \begin{macro}{\TODO}
%    やるべきことを\TODO{強調表示}する
%    \begin{macrocode}
\newcommand{\TODO}[1]{\textcolor{red}{[TODO:#1]}}
%    \end{macrocode}
% \end{macro}
%
% \begin{macro}{\QUERY}
%    気になることを\QUERY{色をかえて}表示する
%    \begin{macrocode}
\newcommand{\QUERY}[1]{\textcolor{green}{#1}}
%    \end{macrocode}
% \end{macro}
%
% \begin{macro}{\erf}
%    誤差関数$\erf$
%    \begin{macrocode}
\newcommand{\erf}{\text{erf}}
%    \end{macrocode}
% \end{macro}
%
% \begin{macro}{\R}
%    実数値体$\R$
%    \begin{macrocode}
\newcommand{\R}{\mathbb{R}}
%    \end{macrocode}
% \end{macro}
%
% \begin{macro}{\dif}
% 常微分記号$\dif{x}$を出力する($x$は引数)
%    \begin{macrocode}
\newcommand{\dif}[1]{\frac{d}{d #1 } }
%    \end{macrocode}
% \end{macro}
%
% \begin{macro}{\difn}
% 常微分記号$\difn{x}{n}$を出力する($x$は第1引数、$n$は第2引数)
%    \begin{macrocode}
\newcommand{\difn}[2]{\frac{d^#2}{d #1^#2 } }
%    \end{macrocode}
% \end{macro}
%
% \begin{macro}{\diff}
% 常微分記号$\diff{x}{u}$を出力する($x$は第1引数、$u$は第2引数)
%    \begin{macrocode}
\newcommand{\diff}[2]{\frac{d #2 }{d #1 } }
%    \end{macrocode}
% \end{macro}
% 
% \begin{macro}{\diffn}
% 常微分記号$\diffn{x}{u}{n}$を出力する($x$は第1引数、$u$は第2引数、$n$は第3引数)
%    \begin{macrocode}
\newcommand{\diffn}[3]{\frac{d^#3 #2 }{d #1^#3 } }
%    \end{macrocode}
% \end{macro}
%
% \begin{macro}{\difp}
% 偏微分記号$\difp{x}$を出力する($x$は引数)
%    \begin{macrocode}
\newcommand{\difp}[1]{\frac{\partial}{\partial #1 } }
%    \end{macrocode}
% \end{macro}
%
% \begin{macro}{\difpn}
% 偏微分記号$\difpn{x}{n}$を出力する($x$は引数)
%    \begin{macrocode}
\newcommand{\difpn}[2]{\frac{\partial^#2}{\partial #1^#2 } }
%    \end{macrocode}
% \end{macro}
%
% \begin{macro}{\diffp}
% 偏微分記号$\diffp{x}{u}$を出力する($x$は第1引数、$u$は第2引数)
%    \begin{macrocode}
\newcommand{\diffp}[2]{\frac{\partial #2 }{\partial #1 } }
%    \end{macrocode}
% \end{macro}
% 
% \begin{macro}{\diffpn}
% 偏微分記号$\diffpn{x}{u}{n}$を出力する($x$は第1引数、$u$は第2引数)
%    \begin{macrocode}
\newcommand{\diffpn}[3]{\frac{\partial^#3 #2 }{\partial #1^#3 } }
%    \end{macrocode}
% \end{macro}
% 
% \begin{macro}{\FourierChebyshev}
% Fourier-Chebyshev展開$$\FourierChebyshev{\phi}$$
%    \begin{macrocode}
\newcommand{\FourierChebyshev}[1]{\sum_{n,m = 0}^{\infty} #1_{nm} \text{e}^{ik_nx} T_m (y) + c.c.}
%    \end{macrocode}
% \end{macro}
%
% \begin{macro}{\FourierChebyshevCos}
% Fourier-Chebyshev展開(cos表現)$$\FourierChebyshevCos{\phi}$$
%    \begin{macrocode}
\newcommand{\FourierChebyshevCos}[1]{\sum_{n=0}^{\infty} \text{e}^{ik_nx} \left( \frac{#1_{n0}}{2} + \sum_{m=1}^{\infty} #1_{nm} \cos(m\theta) \right) + c.c.}
%    \end{macrocode}
% \end{macro}
%
% \Finale
